\documentclass[12pt]{article}
\usepackage{amsmath}
\usepackage{hyperref}
\usepackage{verbatim}
\usepackage{graphicx}
\begin{document}
At any point in space and time, we can evaluate the strength of the magnetic field experienced by a particle as a constant, and construct a basis such that the velocity is broken up into a direction perpendicular to the magnetic field line, and parallel to the magnetic field line. In other words, we can construct a basis for a particle such that $\vec{v}=v_x\hat{x}+v_y\hat{y}+v_z\hat{z}$ and $\vec{B}=B\hat{z}$. Plugging these vectors into the Lorentz force equation we get
\begin{equation}\label{lorentzgyrovec}
\vec{F}=q\left(v_yB\hat{x}-v_xB\hat{y}\right)
\end{equation}
Which can be written in the form of three coupled ordinary differential equations
\begin{equation}\label{lorentzgyro}
\begin{split}
\frac{dv_x}{dt}=\frac{qB}{m}v_y\rightarrow\frac{d^2v_x}{dt^2}=-\frac{q^2B^2}{m^2}v_x\\
\frac{dv_y}{dt}=-\frac{qB}{m}v_x\rightarrow\frac{d^2v_y}{dt^2}=-\frac{q^2B^2}{m^2}v_y\\
\frac{dv_z}{dt}=0
\end{split}
\end{equation}
Recognizing these as simple harmonic oscillator equations, we can define the gyrofrequency \cite{Stacey}
\begin{equation}\label{gyrofrequency}
\omega_B=\frac{qB}{m}
\end{equation}
We can find the following arbitrary initial conditions
\begin{equation}\label{initialcond}
\begin{split}
v_x(0)=v_{x0}\hspace{24mm}v_y(0)=v_{y0}\\
\frac{dv_x}{dt}(0)=\omega_Bv_{y0}\hspace{5mm}\frac{dv_y}{dt}(0)=-\omega_Bv_{x0}
\end{split}
\end{equation}
Solving \eqref{lorentzgyro} we get
$$v_x(t)=v_{x0}\cos(\omega_B)+v_{y0}\sin(\omega_Bt)$$
$$v_y(t)=v_{y0}\cos(\omega_B)-v_{x0}\sin(\omega_Bt)$$
We can put this in amplitude form to get
\begin{equation}\label{velocities}
\begin{split}
v_x=A^*\cos(\omega_Bt+\phi)\\
v_y=-A^*\sin(\omega_Bt+\phi)\\
A^*=\sqrt{v_{x0}^2+v_{y0}^2}=v_{\perp}\hspace{5mm}\phi\equiv\arctan\left(\frac{v_{y0}}{v_{x0}}\right)
\end{split}
\end{equation}
Integrating \eqref{velocities} we get the parametric equations for a circle of radius $r_g$ centered at the point $(x_0,y_0)$
\begin{equation}\label{positions}
\begin{split}
x(t)=r_g\sin(\omega_Bt+\phi)+x_0\\
y(t)=r_g\cos(\omega_Bt+\phi)+y_0
\end{split}
\end{equation}
Where we have defined the gyroradius as \cite{Stacey}
\begin{equation}\label{gyroradius}
r_g=\frac{A^*}{\omega_B}=\frac{mv_{\perp}}{qB}
\end{equation}
In practice, we initialize particles in a magnetic mirror with \eqref{positions} with different phases $\phi$ so that they start distributed about the same point. As well, since drifts in magnetic mirrors act only in the $\hat{\theta}$ direction as discussed in \ref{subsecdrifts}, the guiding center (the point which is at the center of the gyro-orbit) does not change over time. As this is the case, we can use \eqref{gyroradius} to determine the current which should be run through the coils at either end of the mirror. This can be done because at the coil we do not want the particle's radial position to be larger than the radius of the coil, so \eqref{gyroradius} can be solved for field strength, and current can then be solved for from \eqref{fullcoilBr} and \eqref{fullcoilBz}.
\end{document}