\documentclass[12pt]{article}
\usepackage{amsmath}
\usepackage{hyperref}
\usepackage{verbatim}
\usepackage{graphicx}
\begin{document}
Taken primarily from Weston Stacey's Introduction to Fusion book
\subsubsection{$\vec{E}\wedge\vec{B}$ Drift}
If we add an electric field to the system which is perpendicular to the magnetic field it will accelerate particles traveling on their gyro orbits. This prevents the orbits from closing because at somepoint in the orbit the particle will be fighting the $E$-field, and at another point, it will be being pushed by it. This yields a drift perpendicular to the magnetic field, as orbiting through the electric field accelerates and decelerates an equal amount, if we average over the gyro-orbit the overall acceleration is zero, an so we can write
$$\vec{0}=\vec{E}+\vec{v}_{E\wedge B}\wedge\vec{B}$$
This is the same as looking at the force balance on the gyrocenter.
If we take the lefthanded cross product of $\vec{B}$ on either side of the equation we obtain
$$\vec{0}=\vec{B}\wedge\vec{E}+\vec{B}\wedge\vec{v}_{E\wedge B}\wedge\vec{B}$$
We can then use the vector triple product, and the fact that $\vec{v}_{E\wedge B}$ is perpendicular to $\vec{B}$ to obtain
\begin{equation}\label{EcrossB}
\vec{v}_{E\wedge B}=\frac{\vec{E}\wedge\vec{B}}{B^2}
\end{equation} 
For our purposes, we want $\vec{E}\perp\vec{B}$, so we can find the magnitude of the drift velocity to be
\begin{equation}\label{driftmag}
v_{E\wedge B}=\frac{E}{B}
\end{equation} 
Of course, normally, we look at drifts in Cartesian coordinates, so the drift just goes off in a straight line, but in cylindrical coordinates, it is possible to isolate the $\hat{\theta}$ basis vector, and thus produce a rotational plasma. But how do we do this? First, we see that the direction of the drift is dictated by $\vec{E}\wedge\vec{B}$, so we must force this cross product to be only in the $\hat{\theta}$ direction, or else particles will leave the system. This can be accomplished by ensuring that the $E$-field does not have a $\hat{\theta}$ component, as the $B$-field for a magnetic mirror only has $\hat{r}$ and $\hat{z}$ components, this yields a cross product of $(B_rE_z-E_rB_z)\hat{\theta}$, so that answers the directional question. However, we want the electric field to be perpendicular to the magnetic field so that we can get predictable results. This can be accomplished systematically by taking a dot product of the two fields and setting it to zero, which yields:
\begin{equation}\label{radialE}
E_r=-\frac{B_z}{B_r}E_z
\end{equation}
As an electric field is already built into the Boris algorithm, all we need to do to implement this is to define the electric field according to \eqref{radialE}.
\subsubsection{$\nabla B$ Drift}
A charged particle traveling in a gyro-orbit produces a loop of current which has a magnetic moment,$\mu$, which is the first adiabatic invariant, and thus assumed to be conserved in most cases. The force on the magnetic moment (not the particle, but the magnetic moment itself) is given as
$$\vec{F}=-\mu\nabla B$$
This form assumes that $\mu$ is conserved. If we look at the force balance on the magnetic moment (the gyrocenter) we get:
$$0=-\mu\nabla B+e(\vec{v}_{\nabla B}\wedge\vec{B})$$
We can then take advantage of the vector triple product again to obtain:
\begin{equation}\label{gradB}
\vec{v}_{\nabla B}=\frac{mv_{\perp}^2}{2e}\frac{\vec{B}\wedge\nabla B}{B^3}
\end{equation}
The force that produces this drift is only acting on the gyrocenter, so a full orbit calculation accounts for this effect without needing to add an additional force to the particle's Lorentz force equation.
\subsubsection{Curvature Drift}
From classical mechanics we know that a particle traveling along a curved arc with radius of curvature R experiences the fictitious "centrifugal" force:
$$\vec{F}=-\frac{mv_{\parallel}^2}{R}\vec{n}_c$$
Acting towards the center of curvature, $\vec{n}_c$ is a unit vector in the direction of the radius of curvature. If we look at the force balance on the gyrocenter:
$$0=-\frac{mv_{\parallel}^2}{R}\vec{n}_c+e(\vec{v}_{c}\wedge\vec{B})$$
Doing the vector triple product:
\begin{equation}\label{Cuvature}
\vec{v}_c=-\frac{mv_{\parallel}^2}{eR}\frac{\vec{B}\wedge\vec{n}_c}{B^2}
\end{equation}
Note that in a magnetic mirror, all of these drifts are in the $\hat{\theta}$ direction, so containment is not breached.
\subsubsection{Charge Separation}
The curvature drift and the $\nabla B$ drift are charge dependent, so they produce a current density:
\begin{equation}\label{current}
\vec{j}=n_ie_i\vec{v}_i+n_ee_e\vec{v}_e
\end{equation}
Where $n$ is the density, subscript $i$ indicates ions, and subscript $e$ indicates electrons. This creates charge separation which then creates an electric field which is perpendicular to $\vec{B}$ and thus creates an $E\wedge B$ drift.
\end{document}